\documentclass[a4paper, 11pt]{article}
\usepackage[utf8]{inputenc}
\usepackage[T1]{fontenc}

\usepackage[margin=1in]{geometry}
\usepackage{amsmath, amsfonts, amsthm, amssymb, amsxtra}

\usepackage{graphicx}
\usepackage{float}

\usepackage[dvipsnames]{xcolor}

\usepackage{tabularray}
\usepackage{enumitem}
\usepackage{multicol}

\usepackage{hyperref}
\hypersetup{hidelinks}

\usepackage[sfdefault]{notomath}

\usepackage{tikz}
\usetikzlibrary{intersections, angles, calc, positioning}
\usetikzlibrary{shapes.geometric, arrows.meta}
\usetikzlibrary{decorations.pathmorphing, decorations.pathreplacing}

\setlength{\parindent}{0pt}
\setlength{\parskip}{5pt}

\begin{document}
\begin{center}
    \fbox{\LARGE\scshape Epidemia Zumbi}
\end{center}

\section{Explicação do problema}
Por meio de equações diferenciais, é possível montar um modelo matématico para entender como uma população varia em um apocalipse zumbi, para isso iremos considerar três grupos
\begin{itemize}[label=-]
    \item Humanos (H)
    \item Zumbis (Z)
    \item Removidos [mortos que podem retornar como zumbis] (R)
\end{itemize}
as funções, $H(t)$, $Z(t)$ e $R(t)$ representam a população de cada grupo em dado instante de tempo.

Nesse modelo também estamos considerando três parametros positivos
\begin{itemize}[label=-]
    \item $\alpha$ relacionado às interações humano-zumbi que removem zumbis
    \item $\beta$ relacionado às interações humano-zumbi que transformam humanos em zumbis
    \item $\zeta$ relacinado aos removidos que voltam como zumbis
\end{itemize}

O modelo funciona da seguinte forma, Uma interação entre um humano e um zumbi pode resultar em
\begin{itemize}[label=-]
    \item O humano sendo transformado em zumbi
    \item O zumbi sendo removido
\end{itemize}
Além disso, apenas o grupo de humanos pode ser infectado por zumbis.
Também é importante notar que nesse modelo não estamos considerando outras possíveis causas de morte em humanos e também estamos desconsiderando o nascimento de novos humanos, pois isso faria com que os zumbis tivessem uma fonte ilimitada de humanos para serem transformados em zumbis

Assim, temos que o modelo é dado por
\begin{align*}
    H'(t) &= -\beta H(t) Z(t)\\
    Z'(t) &= \beta H(t) Z(t) + \zeta R(t) - \alpha H(t) Z(t)\\
    R'(t) &= \alpha H(t) Z(t) - \zeta R(t)
\end{align*}

\section{Consistencia}

[explicar definição de consistencia olhar aula 9]

Vamos analisar a consistencia utilizando o método de Euler, primeiramente vamos discretizar o sistema

\begin{align*}
    \frac{H(t_{n+1}) - H(t_n)}{h} &= -\beta H(t_n) Z(t_n)\\
    \frac{Z(t_{n+1}) - Z(t_n)}{h} &= \beta H(t_n) Z(t_n) + \zeta R(t_n) - \alpha H(t_n) Z(t_n)\\
    \frac{R(t_{n+1}) - R(t_n)}{h} &= \alpha H(t_n) Z(t_n) - \zeta R(t_n)
\end{align*}

e isolando os termos no instante de tempo $t_{n+1}$

\begin{align*}
    H(t_{n+1}) &=  H(t_n)  + h[-\beta H(t_n) Z(t_n)]\\
    Z(t_{n+1}) &=  Z(t_n) + h[\beta H(t_n) Z(t_n) + \zeta R(t_n) - \alpha H(t_n) Z(t_n)]\\
    R(t_{n+1}) &= R(t_n) + h[\alpha H(t_n) Z(t_n) - \zeta R(t_n)]
\end{align*}

Note que $t_{n+1} = t_n + h$, então podemos fazer uma expansão em Taylor para as equações no tempo $t_{n+1}$

\begin{align*}
    H(t_n + h) &= H(t_n) + H'(t_n) h + H''(t_n) \tfrac{h^2}{6} + \mathcal O (h^3)\\
    Z(t_n + h) &= Z(t_n) + Z'(t_n) h + Z''(t_n) \tfrac{h^2}{6} + \mathcal O (h^3)\\
    R(t_n + h) &= R(t_n) + R'(t_n) h + R''(t_n) \tfrac{h^2}{6} + \mathcal O (h^3)\\
\end{align*}

\begin{align*}
    H(t_n + h) - H(t_n) - H'(t_n) h&=  H''(t_n) \tfrac{h^2}{6} + \mathcal O (h^3)\\
    Z(t_n + h) - Z(t_n) + Z'(t_n) h&= Z''(t_n) \tfrac{h^2}{6} + \mathcal O (h^3)\\
    R(t_n + h) - R(t_n) + R'(t_n) h&=  R''(t_n) \tfrac{h^2}{6} + \mathcal O (h^3)\\
\end{align*}

% \begin{align*}
%     H(t_n + h) - [H(t_n)  + h[-\beta H(t_n) Z(t_n)]] &= H(t_n) + H'(t_n) h + H''(t_n) \tfrac{h^2}{6} + \mathcal O (h^3)\\
%      &- [H(t_n)  + h[-\beta H(t_n) Z(t_n)]]\\
%      Z(t_n + h) -[Z(t_n) + h[\beta H(t_n) Z(t_n) + \zeta R(t_n) - \alpha H(t_n) Z(t_n)]] &= Z(t_n) + Z'(t_n) h + Z''(t_n) \tfrac{h^2}{6} + \mathcal O (h^3)\\ 
%      &-[Z(t_n) + h[\beta H(t_n) Z(t_n) + \zeta R(t_n) - \alpha H(t_n) Z(t_n)]]\\
%      R(t_n + h) -[R(t_n) + h[\alpha H(t_n) Z(t_n) - \zeta R(t_n)]] &= R(t_n) + R'(t_n) h + R''(t_n) \tfrac{h^2}{6} + \mathcal O (h^3)\\
%      &-[R(t_n) + h[\alpha H(t_n) Z(t_n) - \zeta R(t_n)]]\\
% \end{align*}

\begin{align*}
    \frac{H(t_n + h) - H(t_n)}{h} - H'(t_n) &=  \mathcal{O}(h)\\
    \frac{Z(t_n + h) - Z(t_n)}{h} - Z'(t_n) &= \mathcal{O}(h)\\
    \frac{R(t_n + h) - R(t_n)}{h} - R'(t_n) &=  \mathcal{O}(h)\\
\end{align*}

Assim, temos que a forma explicita desse problema, com o método de Euler é consistente com ordem de consitencia linear $\mathcal O (h)$.

Note que quando o passo $h$ tende a 0, a forma discretizada se aproxima da derivada




\end{document}